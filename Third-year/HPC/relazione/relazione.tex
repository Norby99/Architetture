\documentclass[a4paper,12pt, oneside]{article}
\title{Progetto di High Performance Computing}
\author{Gabos Norbert 0000970451}
\date{\today}

\usepackage[T1]{fontenc}
\usepackage[utf8]{inputenc}
\usepackage[italian]{babel}
\pagestyle{plain}
\usepackage{graphicx}
\graphicspath{{images/}}

\begin{document}

\maketitle

\section{Introduzione}

Il programma SPH si occupa di simulare il comportamento di fluidi in un ambiente virtuale.
Il programma è scritto in C ed è stato ideato per essere eseguito su un singolo processore.
Nelle sezioni successive verranno presentate le varie versioni del programma, con le relative
modifiche e le performance ottenute con l'uso di OpenMP e MPI.
Per poter realizzare le versioni parallele del programma si è dovuto analizzare la versione
sequenziale, per capire quali sono le parti del codice che possono essere eseguite in parallelo.

\section{Versione OpenMP}

Per creare la versione OpenMP, si è partiti dalla versione sequenziale analizzando il codice.
Una volta individuati i punti da parallelizzare si sono anallizzate eventuali loop carry dependencies.
La funzione più semplice da parallelizzare è stata la integrate in quanto consiste in un singolo ciclo.
Per le funzioni compute_density_pressure e compute_forces si è deciso di parallelizzare il ciclo
\begin{itemize}
\item compute_density_pressure 
\item compute_forces
\item integrate
\item avg_velocities
\end{itemize}


\section{Versione MPI}
\section{Conclusioni}

\end{document}
