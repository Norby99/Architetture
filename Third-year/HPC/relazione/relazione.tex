\documentclass[a4paper,12pt, oneside]{article}
\title{Progetto di High Performance Computing}
\author{Gabos Norbert 0000970451}
\date{\today}

\usepackage[T1]{fontenc}
\usepackage[utf8]{inputenc}
\usepackage[italian]{babel}
\pagestyle{plain}
\usepackage{graphicx}
\graphicspath{{images/}}

\begin{document}

\maketitle

\section{Introduzione}

Il programma SPH si occupa di simulare il comportamento di fluidi in un ambiente virtuale.
Il programma è scritto in C ed è stato ideato per essere eseguito su un singolo processore.
Nelle sezioni successive verranno presentate le varie versioni del programma, con le relative
modifiche e le performance ottenute con l'uso di OpenMP e MPI.
Per poter realizzare le versioni parallele del programma si è dovuto analizzare la versione
sequenziale, per capire quali sono le parti del codice che possono essere eseguite in parallelo.

\section{Versione OpenMP}

Per creare la versione OpenMP, si è partiti dalla versione sequenziale analizzando il codice.
Una volta individuati i punti da parallelizzare si sono anallizzati eventuali loop carry dependencies
e si è scoperto che tutti i cicli erano embarrassingly parallel.
Si è partiti dalla funzione più semplice da parallelizzare, ovvero la "integrate" in quanto consiste in un
singolo ciclo. Le funzioni compute_density_pressure e compute_forces contengono al loro interno due cicli
annidati, quindi hanno richiesto un po' più di attenzione.
Per la funzione "compute_density_pressure" l'intento era quello di parallelizzare entrambi i cicli mediante
la clausola "collapse(2)", però i due cicli sono dipendendi uno dall'altro, quindi si è deciso di
parallelizzare solo il ciclo esterno.
Per la funzione "compute_forces" l'aproccio è stato quello di parallelizzare il ciclo interno eseguendo
una "reduce" sulle variabili fpress_x, fpress_y, fvisc_x e fvisc_y. Questa soluzione purtroppo è stata
scartata in quanto non ha portato ad un miglioramento delle prestazioni.
Infine per la funzione "avg_velocities" si è deciso di utilizzare la funzione reduce sulla variabile
"result" per ottenere il risultato finale.

\section{Versione MPI}
\section{Conclusioni}

\end{document}
