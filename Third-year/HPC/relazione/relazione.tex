\documentclass[a4paper,12pt, oneside]{article}
\title{Progetto di High Performance Computing}
\author{Gabos Norbert 0000970451}
\date{\today}

\usepackage[T1]{fontenc}
\usepackage[utf8]{inputenc}
\usepackage[italian]{babel}
\pagestyle{plain}
\usepackage{graphicx}
\graphicspath{{images/}}

\begin{document}

\maketitle

\section{Introduzione}

Il programma SPH è un simulatore che si occupa di modellare il comportamento dei fluidi in un
ambiente virtuale. È stato scritto in linguaggio C e progettato per funzionare su un singolo
processore. Per realizzare le versioni parallele del programma, è stato necessario analizzare
la versione sequenziale, al fine di individuare le parti del codice che possono essere eseguite
in parallelo. Le sezioni successive presentano le diverse versioni del programma, con le
relative modifiche e le performance ottenute grazie all'utilizzo di OpenMP e MPI.

\section{Versione OpenMP}

Per creare la versione OpenMP, si è partiti dalla versione sequenziale analizzando il codice.
Una volta individuati i punti da parallelizzare, si sono analizzate eventuali loop carry
dependencies, e si è scoperto che tutti i cicli erano embarrassingly parallel. Si è partiti
dalla funzione più semplice da parallelizzare, ovvero "integrate", in quanto consiste in un
singolo ciclo. Le funzioni compute_density_pressure e compute_forces contengono al loro
interno due cicli annidati, quindi hanno richiesto un po' più di attenzione.
Per la funzione "compute_density_pressure", l'intento era quello di parallelizzare entrambi
i cicli mediante la clausola "collapse(2)", però i due cicli sono dipendenti uno dall'altro,
quindi si è deciso di parallelizzare solo il ciclo esterno. Per la funzione "compute_forces",
l'approccio è stato quello di parallelizzare il ciclo interno, eseguendo una "reduce" sulle
variabili fpress_x, fpress_y, fvisc_x e fvisc_y. Questa soluzione, purtroppo, è stata scartata
in quanto non ha portato ad un miglioramento delle prestazioni.
Infine, per la funzione "avg_velocities", si è deciso di utilizzare la funzione reduce sulla
variabile "result" per ottenere il risultato finale.

% manca la tabella con i tempi e la descrizione dei risultati

\section{Versione MPI}

Durante l'inizializzazione della versione MPI è stata usata la funzione Bcast, per distribuire le
informazioni delle particelle tra i processori. Inoltre è stata creato il tipo di variabile
"MPI_PARTICLE" per poter inviare e ricevere le particelle tra i processori.

Per parallelizzare le varie funzioni l'aproccio è stato quello di fare un modo che ogni processo
avesse le stesse particelle sia all'inizio che alla fine della funzione. Per fare ciò si è
creata la funzione "sync_particles", che mediante la funzione "MPI_Allgatherv" permette di
raccogliere i vari dati delle particelle sparse su tutti i processori e di riorganizzarli in
modo che ogni processo abbia le stesse particelle. La funzione "sync_particles" viene richiamata
alla fine di ogni funzione. Per parallelizzare tutti le funzione è stato utilizzato un aproccio
simile, ovvero si è diviso l'array in chunk ed ogni processo ha eseguito solo il proprio chunk.
Per quanto riguarda la funzione "avg_velocities", si è deciso di utilizzare la funzione "reduce"
per ottenere il risultato finale e salvarlo tra tutti i processi, questa cosa si poteva risolvere anche
solo con la funzione "MPI_Reduce", ma si è deciso di utilizzare la "MPI_Allreduce" per avere
una soluzione più generale.

\section{Conclusioni}

\end{document}
