\documentclass[a4paper,12pt, oneside]{article}

\title{\textbf{L'impatto del Codec AV1 sull'industria della visualizzazione online} \\ \large A.A 2023/2024 \\ Elaborato di Crittografia}
\author{Gabos Norbert \\ 0000970451 \\ tiberiunorbert.gabos@studio.unibo.it }
\date{}

\usepackage[T1]{fontenc}
\usepackage[utf8]{inputenc}
\usepackage[italian]{babel}
\pagestyle{plain}
\usepackage{graphicx}
\usepackage[table,xcdraw]{xcolor}
\usepackage{tabularx}
\usepackage{ragged2e}               % migliora la formattazione del testo all'interno delle celle
\renewcommand{\arraystretch}{1.5}   % aggiunge margine alle celle
\graphicspath{{images/}}

\definecolor{darkBlue}{RGB}{21, 69, 179}
\definecolor{lightPink}{RGB}{242, 10, 172}

\begin{document}

\maketitle

\newpage
\tableofcontents{}
\newpage

\section{Introduzione}

\section{Tecnologie attuali}
\subsection{H.264}
\subsection{VP9}    % forse non serve descriverlo
\subsection{AV1}
\subsection{Comparazione delle prestazioni}

\section{Vantaggi per le aziende}

\section{Sfide e considerazioni}
Non dimenticare di discutere anche delle sfide e delle considerazioni pratiche legate all'adozione di nuovi codec, come il supporto hardware/software, la compatibilità con dispositivi esistenti e la gestione dei diritti d'autore.

\section{Conclusioni}

\section{Bibliografia}

\end{document}
