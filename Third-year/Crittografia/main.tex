\documentclass[a4paper,12pt, oneside]{article}

\title{\textbf{L'impatto del Codec AV1 sull'industria della visualizzazione online} \\ \large A.A 2023/2024 \\ Elaborato di Crittografia}
\author{Gabos Norbert \\ 0000970451 \\ tiberiunorbert.gabos@studio.unibo.it }
\date{}

\usepackage[T1]{fontenc}
\usepackage[utf8]{inputenc}
\usepackage[italian]{babel}
\pagestyle{plain}
\usepackage{graphicx}
\usepackage[table,xcdraw]{xcolor}
\usepackage{tabularx}
\usepackage{ragged2e}               % migliora la formattazione del testo all'interno delle celle
\renewcommand{\arraystretch}{1.5}   % aggiunge margine alle celle
\graphicspath{{images/}}

\definecolor{darkBlue}{RGB}{21, 69, 179}
\definecolor{lightPink}{RGB}{242, 10, 172}

\begin{document}

\maketitle

\newpage
\tableofcontents{}
\newpage

\section{Introduzione}

\subsection{Prime idee}
\textbf{Intraframe}: L'intraframe compression, o compressione intra-frame, si riferisce alla
compressione dei dati all'interno di un singolo fotogramma. Questo metodo sfrutta le ridondanze
spaziali all'interno di un singolo frame.
\\\\\textbf{Interframe}: L'interframe compression, o compressione inter-frame, lavora sulla
ridondanza temporale tra i frame adiacenti. Invece di codificare ogni frame singolarmente,
vengono identificate e codificate solo le differenze tra i frame successivi o precedenti.
Questo consente di ridurre ulteriormente le dimensioni del file video, poiché molte
informazioni rimangono costanti tra i fotogrammi vicini.
\\\\
TODO: Magari si potrebbero spiegare meglio queste due tecniche più avanti nell'articolo, quindi sarebbe meglio scrivere in questo
TODO: punto che questi due algoritmi verranno spiegati meglio nei paragrafi successivi

\subsection{H.261}
Nel 1988 nasce il codec H261 che è stato il primo algoritmo di compressione video ad utilizzare
in modo efficiente degli algoritmo di \textbf{intraframe} e \textbf{interframe} ed è
responsabile dell'introduzione della codifica video ibrida basata su blocchi, che è ancora
utilizzata oggi in molti standard video.

\section{Tecnologie attuali}
\subsection{H.264 e H.265}
TODO: L'h264 essendo una tecnica che racchiude molti algoritmi complessi per la compressione video, necessita di accelleratori hardware
TODO: per poter decodificare i fotogrammi in tempo reale.
\subsection{VP9}    % forse non serve descriverlo
\subsection{AV1}
\subsection{Comparazione delle prestazioni}

\section{Vantaggi per le aziende}

\section{Sfide e considerazioni}
Non dimenticare di discutere anche delle sfide e delle considerazioni pratiche legate all'adozione di nuovi codec, come il supporto hardware/software, la compatibilità con dispositivi esistenti e la gestione dei diritti d'autore.

\section{Conclusioni}

\section{Bibliografia}
sito ufficiale di h.265 https://hevc.hhi.fraunhofer.de/
repository di h.265 https://vcgit.hhi.fraunhofer.de/jvet/HM
sito che parla della storia degli encorder https://api.video/blog/video-trends/the-history-of-video-compression-starts-in-1929/
utile! parla dei tecnicismi del h.264 nella sezione di Implementing h.264 https://www.embedded.com/implementing-h-264-video-compression-algorithms-on-a-software-configurable-processor/
utile! parla dei tecnicismi del h.265 nella sezione di Coding tools https://en.wikipedia.org/wiki/High_Efficiency_Video_Coding

\end{document}
